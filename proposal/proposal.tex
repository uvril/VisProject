\documentclass{article}
\title{CS6630 Visualization Fall 2017 Project Proposal\\ A Mirror of History}
\author{Yanqing Peng, Yuwei Wang \\ u1076076, u1140944 \\ yanqing.utah@gmail.com, yuwei.utah@gmail.com}
\begin{document}
\maketitle
\section{Background and Motivation}

\section{Project Objectives}
The objective of this project is to visualize the historical rise and fall.
We would like to explore the following points:
\begin{itemize}
    \item {\emph What was the world look like in the past?} To answer that, we would like to visualize the borders of countries at different times.
    \item {\emph How countries develop over time?} To answer that, we would like to visualaize some statistical data for countries, e.g., population, GDP, etc.
    \item {\emph How past events change the world?} To answer that, we would like to maintain a list of events, and visualize their impacts on the country borders/country statistics/other visualized data.
    \item Other historical trends that we are interested. We leave these for optional features. Some possible candidates include:
        \begin{itemize}
            \item 
        \end{itemize}

\end{itemize}
\section{Data}
Historical data are easy to retrieve since they are public to all. Useful resources include:
\begin{itemize}
    \item Wikipedia, for statistical data (e.g. population).
    \item Thenmap.net, generates historical maps with GeoJSON/TopoJSON format.
    \item Strategy games about history, (e.g. Crusader King, Europa Universalis, Heart of Irons, Victoria). Although the data from strategy games are not accurate, they can be use
        as a reference.
    \item Educational resources and other resources. Again, historical data are public to all, so we can eventually find all we want.
\end{itemize}

\section{Data Processing}

\section{Visualization Design}
\section{Must-Have Features}
\section{Optional Features}
\section{Project Schedule}
\end{document}
