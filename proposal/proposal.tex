\documentclass[12pt, fullpage,letterpaper]{article}
\usepackage[margin=1in]{geometry}
\title{CS6630 Visualization Fall 2017 Project Proposal\\ A Mirror of History}
\author{Yanqing Peng, Yuwei Wang}
\begin{document}
\maketitle

\section{Basic Information}
\paragraph{Title} A Mirror of History
\paragraph{Group members}
\begin{itemize}
    \item Yanqing Peng, u1076076, yq.peng@utah.edu
    \item Yuwei Wang, u1140944, yuwei.utah@gmail.com
\end{itemize}
\paragraph{Github link} https://github.com/uvril/VisProject

\section{Background and Motivation}

History is always fascinating.
It's important to learn lessons from the past, by understanding the causality behind historical events.
To learn about history is to learn about humanity, which basically means
developing an understanding of ourselves.
Therefore, knowledge of history helps us prepare better for future.

We've learned a lot of fragmented
knowledge from books and the Internet, but they've never been
put together in a conherent manner.
The course project of CS6630 gives us a golden opportunity to
learn better about history.
We are willing to make a deeper
insight into history, with the help of tools we've learned from this class.

\section{Project Objectives}
The objective of this project is to visualize the historical rise and fall.
We would like to explore the following points:
\begin{itemize}
    \item {\emph What was the world look like in the past?} To answer that, we would like to visualize the borders of countries at different times.
    \item {\emph How countries develop over time?} To answer that, we would like to visualaize some statistical data for countries, e.g., population, GDP, etc.
    \item {\emph How past events change the world?} To answer that, we would like to maintain a list of major historical events, and visualize their impacts on the country borders, country statistics or other visualized data.
\end{itemize}
\section{Data}
Historical data are easy to retrieve since they are public to all. Useful resources include:
\begin{itemize}
    \item Wikipedia, for statistical data (e.g. population).
    \item Thenmap.net, generates historical maps with GeoJSON/TopoJSON format.
    \item Strategy games about history, (e.g. Crusader King, Europa Universalis, Heart of Irons, Victoria). Although the data from strategy games are not accurate, they can be use
        as a reference.
    \item Educational resources and other resources (e.g. Chronas.org). There are plenty of educational resourses that can help us.
\end{itemize}

\section{Data Processing}
Since there are so many candidate data sources, we are unable to make it clear how to do data processing for now.
Basically, we wish to get the following datasets after data processing:
\begin{itemize}
    \item GeoJSON/TopoJSON file for country borders at different time.
    \item A CSV data for each contry to record its statistics (GDP/population/total area etc.) at different time.
    \item A data set contains major historical events and their impacts.
\end{itemize}

We plan to use Python to pre-process the data into desired format before feeding it to Javascript.

\section{Visualization Design}

The visualization should contain at least four charts: a map chart for the world map, a year chart for selecting a desired year, a information panel that displays the information
of selected country at selected year, and finally an aggregation view chart that show some desired aggregation data.

There are two needs for the visualization, which is the need of  ``showing the world at a specific year'' and the need of ``showing the shift of some statistics over years for a specific country''.
To fulfill these two needs nicely, we would like to integrate the year chart, the map chart and the aggregation view chart into a coherent way so that the user can interact with them easily.

\paragraph{Alternative 1}
As shown in Figure \ref{}. 
This is a quite straight-forward way.
Users simply click on the map to select the interested country,
and brush the year chart to select a year range.
The aggregation view chart then shows the data based on the selections.

A major drawback of this design is that, sometimes we're hard to select a desired country.
First, there are some countries that are too small to select, e.g., Liechtenstein. Second, sometimes we are interested to select
a country, but we don't know where it is, e.g., Palau, which is a famous country but most people can't tell where it is.

\paragraph{Alternative 2}
As shown in Figure \ref{}.
We place a standalone aggregation view that we can brush a year range and check a set of countries, and
show a linechart of some desired statistics.

\paragraph{Alternative 3}
As shown in Figure \ref{}.

\paragraph{Final Decision}




\section{Must-Have Features}
\begin{itemize}
    \item An interactive year chart to choose a desired year. All other contents are based on the selected year.
    \item A interactive map that we can select countries. When we select a country, a detailed view containing the data of that country should be shown.
    \item A country list that contains all countries and their statistics. We are able to sort on any attributes so that we can check the largest country, the least developed country, etc.
\end{itemize}
\section{Optional Features}
\begin{itemize}
    \item A list of major historical events. It should be linked to other related views so that we can see the impact of this event.
    \item More available data visualization on the map. In addition to show the country borders, we can further add, say, a heat map of population. It makes us easier to learn the density of population for different parts of the world.
\end{itemize}
\section{Project Schedule}
\paragraph{Oct 30-Nov 5} Start data collection and data processing.
\paragraph{Nov 6-Nov 12} Finish data processing and webpage construction. A simple demo should be made before milestone.
\paragraph{Nov 13-Nov 19} Complete must-have features.
\paragraph{Nov 20-Nov 26} Complete optional features.
\paragraph{Nov 27-Dec 1} Finalize the project.
\end{document}
