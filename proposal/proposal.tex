\documentclass[12pt, fullpage,letterpaper]{article}
\usepackage[margin=1in]{geometry}
\title{CS6630 Visualization Fall 2017 Project Proposal\\ A Mirror of History}
\author{Yanqing Peng, Yuwei Wang}
\begin{document}
\maketitle

The title \textbf{A mirror of history} comes from the epitaph of Wei Zhen,
a famous Chinese historian:

\emph{``Using copper as a mirror allows one to keep his clothes neat. Using
history as a mirror allows one to see the future trends. Using a person as a
mirror allows one to see what is right and what is wrong.''}

\section{Basic Information}
\paragraph{Title} A Mirror of History
\paragraph{Group members}
\begin{itemize}
    \item Yanqing Peng, u1076076, yq.peng@utah.edu
    \item Yuwei Wang, u1140944, yuwei.utah@gmail.com
\end{itemize}
\paragraph{Github link} https://github.com/uvril/VisProject


\section{Background and Motivation}

History is always fascinating.  It's important to learn lessons from the past,
by understanding the causality behind historical events.  To learn about
history is to learn about humanity, which basically means developing an
understanding of ourselves.  Therefore, knowledge of history helps us prepare
better for future.

However, it can be boring to learn history by texts and numbers. We can know
from textbook that the British Empire used to rule 35,500,000 km$^2$ area of
land. But how large is it?  Showing a map of British Empire is much more
interesting than just telling the number.  Similarly, we know that the economy
of both Japan and Singapore grow dramatically after World War II, but both slow
down in 1990s.  How can we present a comparison between the economy development
of Japan and Singapore?  Showing the GDP table will be boring and hard to
understand.  Instead, showing a line chart for the GDP of Japan and Singapore
is much more clearer. Therefore, visualization is essential for learning history.

Our textbooks already contain plenty of figures to visualize history. However,
it's not enough. While learning history, people have different focuses.  Alice
is interested in the border of British Empire in 1913, while Bob wants to see a
map of French colonies in 1908. Charlie would like to know the comparison
between Japan and Singapore, while Eve is curious about the comparison between
India and Pakistan.  Only with interactive visualizations can one develops the
knowledge of history with his own interests.

The course project of CS6630 gives us a golden opportunity to learn better
about history.  We are willing to make a deeper insight into history, by
constructing a interactive visualization on history. We believe that the tools
we've learned from this course will help us fulfill our objective.


\section{Project Objectives}
The objective of this project is to interactively visualize the historical rise
and fall.  We would like to answer the following questions:
\begin{itemize}
    \item Show the historical world map.
    \item For some desired attribute (e.g. population), visualize this attribute on different countries in some specific year.
    \item For a specific country, visualize a desired attribute (e.g. population) over a time range.
    \item For a desired attribute (e.g. population), a set of countries and a set of years, show the aggregation result.
    \item (Optional) Show how historical events change the world.
\end{itemize}

\section{Data}
Historical data are easy to retrieve, since they are actually public data that available to all.
Useful resources include:

\begin{itemize}
    \item Wikipedia, for statistical data (e.g. population).
    \item Thenmap.net, generates historical maps with GeoJSON/TopoJSON format.
    \item Strategy games about history, (e.g. Crusader King, Europa Universalis, Heart of Irons, Victoria). Although the data from strategy games are not accurate, they can be use
        as a reference.
    \item Educational resources and other resources (e.g. Chronas.org). There are plenty of educational resourses that can help us.
\end{itemize}

\section{Data Processing}
Since there are so many candidate data sources, we are unable to make it clear how to do data processing for now.
Basically, we wish to get the following datasets after data processing:
\begin{itemize}
    \item GeoJSON/TopoJSON file for country borders at different time.
    \item A CSV data for each contry to record its statistics (GDP/population/total area etc.) at different time.
    \item A data set contains major historical events and their impacts.
\end{itemize}

We plan to use Python to pre-process the data into desired format before feeding it to Javascript.

\section{Visualization Design}

The visualization should contain at least four parts:

\begin{itemize}
    \item A year chart that we can select a desired year from.
        Basically it should be a row of years, and we can click on a desired year to select it.
        It should support zooming and sliding to help select a specific year.
    \item A map chart that covers the most area of the webpage.
        It shows the world map at the year selected by the year chart.
        A country can be selected with a mouse click.
        We plan to add zooming functionality to make it easier to click on small countries.
    \item A information panel that displays the detailed information of selected country at the selected year.
    \item An aggregation view chart that show some desired aggregation data. We will discuss it below in detail.
\end{itemize}

If we simply want to show the information for some country in some year, the
first three parts are already sufficient.  However, the challenging thing is to
show complicated questions, such as ``show a comparison between the economic of
N. Korea and S. Korea from 1946 to 2012'', or ``show the top 10 countries by 
population in 2017''. Basically, such a question contains a set of countries, a
set of years, and a desired attribute.  We answer these questions in the
aggregation view chart with aggregated data. The design of the aggregation
vie chart and the interact scheme need to be carefully considered.
In what follows, we designed three alternatives for our visualization.
The design of year chart, map chart and information panel is straight-forward,
so we only present them in Alternative 1 for simplicity.

\paragraph{Alternative 1}
As shown in Figure \ref{}. 
This is a quite straight-forward way.
Users simply click on the map to select the interested country,
and brush the year chart to select a year range.
The aggregation view chart then shows the data based on the selections.

A major drawback of this design is that, sometimes we're hard to select a desired country.
First, there are some countries that are too small to select, e.g., Liechtenstein. Second, sometimes we are interested to select
a country, but we don't know where it is, e.g., Palau, which is a famous country but most people can't tell where it is.

\paragraph{Alternative 2}
As shown in Figure \ref{}.
We place a standalone aggregation view that we can brush a year range and check a set of countries, and
show a linechart of some desired statistics.

\paragraph{Alternative 3}
As shown in Figure \ref{}.

\paragraph{Final Decision}




\section{Must-Have Features}
A user should be able to:
\begin{itemize}
    \item Zoom and slide the year chart.
    \item Choose a desired year in the year chart. 
    \item See the world map of the selected year in the map chart.
    \item Select a country in the map chart.
    \item See the detailed information about the selected country in the information panel.
    \item Select several contries and select several years.
    \item Select a desired statistics (e.g. population), show the aggregated data based on the selections in the aggregation view chart.
\end{itemize}

\section{Optional Features}
A user may also be able to:
\begin{itemize}
    \item Zoom the map in order to see small countries.
    \item Select from a list of major historical events. The impact of the selected event is somehow shown on the map chart and the information panel.
    \item Select other views of the map, e.g., a heat map of population. In this case an additional layer of heat map is shown above the map.
\end{itemize}

\section{Project Schedule}
\paragraph{Oct 30-Nov 5} Start data collection and data processing.
\paragraph{Nov 6-Nov 12} Finish data processing and webpage construction. A simple demo should be made before milestone.
\paragraph{Nov 13-Nov 19} Complete most must-have features.
\paragraph{Nov 20-Nov 26} Complete must-have features. If time permits, complete optional features.
\paragraph{Nov 27-Dec 1} Finalize the project.
\end{document}
